\documentclass[border = 2mm]{standalone}
\usepackage{circuitikz}

\begin{document}

\ctikzset{amplifiers/fill=cyan!20, 
            %component text=left
            }
\begin{circuitikz}[scale=1, transform shape]

    \draw (0,0) node[above]{$v_i$} to[short, o-] ++(1,0)
        node[op amp, noinv input up, anchor=+](OA){\texttt{OA1}}
        (OA.-) -- ++(0,-1.5) coordinate(FB)
        to[] ++(0,-2.5) node[ground]{}
        (FB) to[] (FB -| OA.out) -- (OA.out)
        to [short,*-o] ++(1,0) node[above]{$v_o$};
    \node[%
		anchor = south,
		text = gray!90,
		%fill = white,
		align = left,
		font = \large,
    ] (inversorTitle) at 
          ([xshift = 0cm, yshift = 0.75cm]OA.up) 
          {\texttt{Voltage follower OA}};
      \node[%
      anchor = south west,
      text = gray!90,
      %fill = white,
      align = left,
      font = \Large,
    ] (inversorEquation) at 
          ([xshift = 1.25cm, yshift = -1.4cm]FB) 
          {$v_o = v_i$};

\end{circuitikz}

\end{document}
