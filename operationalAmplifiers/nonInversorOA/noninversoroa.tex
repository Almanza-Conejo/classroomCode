\documentclass[border = 1mm]{standalone}
\usepackage{circuitikz}

\begin{document}

\ctikzset{amplifiers/fill=cyan!20, 
            %component text=left
            }
\begin{circuitikz}[scale=1, transform shape]

    %   --  non-inverting AO
    \draw (0, 0) node[op amp, 
        noinv input up, 
        anchor=center,](AOinversor){\texttt{AO}}
        (AOinversor.in up) to[short, -o] ++(-1, 0) node[above]{$v_i$}
        (AOinversor.-) -- ++(0, -1.5) coordinate (ground2)
        to[R=$R_1$] ++(0, -2.5) node[ground]{}
        (ground2) to[R=$R_2$] (ground2 -| AOinversor.out) -- (AOinversor.out)
        to [short, *-o] ([xshift = 1cm]AOinversor.out) node[above]{$v_o$}
        ;
    \node[%
		anchor = south,
		text = gray!90,
		%fill = white,
		align = left,
		font = \large,
	] (inversorTitle) at 
        ([xshift = 0cm, yshift = 0.75cm]AOinversor.up)
        {\texttt{Non-inverting OA}};
    \node[%
		anchor = south west,
		text = gray!90,
		%fill = white,
		align = left,
		font = \Large,
	] (inversorEquation) at 
        ([xshift = 1.25cm, yshift = -1.5cm]ground2) 
        {$v_o = v_i \left(1 + \frac{R_2}{R_1}\right)$};

\end{circuitikz}

\end{document}
